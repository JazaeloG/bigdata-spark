\documentclass{article}
\usepackage{graphicx} 

\title{\textbf{Universidad Veracruzana} }
\date{\textbf{Facultad de Negocios y Tecnologias} }

\begin{document}
\maketitle
%\section
\textsf{\Large \textbf{Experiencia Educativa:} Paradigmas De Programación.\\}
 
\maketitle
%\section
\textsf{\Large \textbf{Alumno:} Omar Jazael Galindo Alducin. \\}

\maketitle
%\section
\textsf{\Large \textbf{Tema:} Reporte Técnico. \\}

\textsf{\ \textbf{Grupo:} 402 ISW 1° Parcial \\}

\maketitle

\textsf{\ Fecha de Entrega: 17 de Marzo del 2023 \\}

\author{JazaeloG}
\date{Marzo 2023}

\newpage

\maketitle
\textbf{Introducción:}\\
\\
Las marcas de tecnología han estado liderando el camino en la creación de soluciones innovadoras para nuestras necesidades tecnológicas, y una de las áreas de mayor impacto es el campo de las redes neuronales de Hopfield. En este reporte técnico, exploraremos la relevancia de este tema y su impacto en la vida cotidiana, con un enfoque en cómo estas redes neuronales utilizan matrices y reconocimiento de vocales mayúsculas para encontrar patrones en los datos. Analizaremos cómo este procedimiento logra encontrar figuras a través de binario en una matriz, y compartiré mi propia experiencia y conclusión personal sobre el tema. Para realizar la identificacion de señales de peligro \\}
\\
\textbf{Las Redes de Hopfield:}
\begin{itemize}
    \item Hopfield conceptualizó las redes neuronales como
        sistemas dinámicos con energía y mostró su
        semejanza con ciertos modelos físicos.

    \item Hopfield propuso varios modelos de redes
recurrentes. En este tipo de redes, la salida de cada
neurón se calcula y se retro-alimenta como entrada,
calculándose otra vez, hasta que se llega a un punto
de estabilidad. 
    \item Supuestamente los cambios en las salidas van
siendo cada vez mas pequeños, hasta llegar a cero,
esto es, alcanzar la estabilidad.
    \item Puede ser que una red recurrente nunca llegue a un
punto estable.   
\end{itemize}
\textsf{P. Gómez Gil. INAOE,(2017)}
\\


\textbf{Configuración de la Red:}
\begin{itemize}
    \item Se utiliza principalmente con entradas binarias.
    \item Se puede utilizar como una memoria asociativa, o para
resolver problemas de optimización.
    \item Una memoria asociativa o dirigida por contenido es
aquella que se puede accesar teniendo una parte de un
patrón de entrada, y obteniendo como resultado el
patrón completo.
    \item Hopfield también utilizó sus redes para resolver un
problema de optimización: el agente viajero. Además
construyó una red con circuitos integrados que convierte
señales analógicas en digitales. 
\end{itemize}
\textsf{P. Gómez Gil. INAOE,(2017)}
\\
\newpage
\textbf{Matriz de representación:}
\begin{itemize}
    \item Asimismo, la matriz se puede representarse
en un vector con N2 elementos, donde N es
el número de ciudades.
    \item A su vez, este vector puede representarse en
una red de Hopfield de N2 neurones.
    \item El objetivo del entrenamiento es hacer
converger la red hacia una ruta válida, en el
cual exista la mínima energía posible.  
\end{itemize}
\textsf{P. Gómez Gil. INAOE,(2017)}
\\



\textbf{Las redes neuronales Hopfield tienen muchas aplicaciones
prácticas en la vida diaria:}
\begin{itemize}
    \item Reconocimiento de patrones: Las redes Hopfield se utilizan en el reconocimiento de patrones, como el reconocimiento de caracteres escritos a mano. Los modelos de redes Hopfield pueden aprender y almacenar patrones, lo que les permite identificar patrones similares en nuevos datos.
    \item Optimización: Las redes Hopfield también se utilizan en problemas de optimización, como la optimización de la distribución de energía o la optimización del enrutamiento de vehículos. Los modelos de redes Hopfield pueden encontrar soluciones óptimas para problemas de optimización.
    \item Modelado de sistemas biológicos: Las redes Hopfield se utilizan para modelar sistemas biológicos, como la memoria y la cognición. Los modelos de redes Hopfield pueden simular la forma en que el cerebro almacena y recupera información.
\end{itemize}
\textbf{Hopfield en Logos de marcas de Tecnología}
\\

Hopfield es un tema importante en el campo de las redes neuronales pero en esta ocasion se tiene en mente la realizacion de hacerla conforme a la identificacion de señales de peligro



\textbf{Conclusión:}

En resumen, Hopfield representa una de las tecnologías más interesantes y prometedoras en el campo de las redes neuronales. Su capacidad para resolver problemas complejos y reconocer patrones tiene implicaciones significativas para el futuro de la tecnología, y las marcas de tecnología líderes están invirtiendo en esta área para impulsar la innovación y el avance tecnológico.



\textbf{Bibliografía:}
\begin{itemize}
    \item P. Gómez Gil. INAOE,(2017).
    \item Bello-Orgaz G., Hernández-Castro J. C., & Camacho D. (2016).
    \item Mirjalili S., Mirjalili S. M., & Lewis A. (2014).
\end{itemize}
\end{document}