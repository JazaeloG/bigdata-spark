\documentclass[10pt]{article}         %% What type of document you're writing.
\usepackage{graphicx}
\usepackage{hyperref}
\usepackage[dvipsnames]{xcolor}

%%%%% Preamble

%% Packages to use

\usepackage{amsmath,amsfonts,amssymb}   %% AMS mathematics macros

%% Title Information.

\title{Red neuronal Simbolos de peligro}
\author{Omar Jazael Galindo Alducin}
%% By default, LaTeX uses the current date

%%%%% The Document

\begin{document}

\maketitle

\begin{abstract}
Este documento implementa red neuronal de simbolos de peligro.
\end{abstract}

\section{Introducción}
El presente proyecto esta diseñado para la realización de una red neuronal, la cual busca por medio de la implementación de cálculos matemáticos y programación en el lenguaje "C++" obtener la probabilidad de que un dato de entrada se asemeje con los datos previamente cargados para entrenar la red neuronal.
En esta ocasión esta red está pensada para simbolos de peligro, los datos de entrenamiento será el abecedario representado en matrices de 7*6 para un mejor resultado de aproximación.
Antes de pasar a la implementación de código se realizará un prototipado con la herramienta Octave que sera sustituto de Matlab, para verificación correcta de los cálculos matemáticos y así obtener los resultados esperados a la hora de ejecutar el programa.

\section{Desarrollo}

Antes de realizar el codigo para comprobar lo que vamos a realizar va a funcionar es necesario realizar un prototipado y para ello se hará uso de la heramienta octave, donde se realizará el prototipo de nuestra red neuronal hopfiel.
Durante el desarrollo de el codigo hubo algunos problemas que inpedian realizar el aprendizaje de la red neuronal y hacia que esta se confundiera con los calores que se le introducian para comparar con los de entrenamiento, como primer inpedimento fue el tamaño de la matriz ya que antes planeaba realizar señales de peligro con matices de 10*10, pero el tamaño era demasiado grande de las matrices , asi que fui incrementando poco a poco la matriz hasta obtener un buen resultado por parte de la red, durante la incrementación de las columnas y filas la red dejaba de funcionar por el segundo problema que fué el groso del simbolo a identificar, es decir, la representacion de 1's, ya que si era muy delgado la red se confundia y mostraba resultados erroneo, asi que probe engruesando el simbolo (letra coreana) y esta dió un buen resultado, sin embargo cabe mencionar que la posicion del simbolo en la matriz de 7*6 tambien afectaba para el reconocimiento de patron, por ello decidi solo introducir 2 valores de entrenamiento para que funcionara correctamente y no se confundiera entre los simbolos o mostrara resultados no existentes.



\section{Conclusión}
Mi conclusión es que el programa funciona correctamente sin embargo tiene algunas limitantes como es el tamaño de la matriz y la posición del la representación de la silueta, es decir, la colocación de los 1's dentro de la matriz, ya que tambien afecta a la hora de mostrar los resultados por la terminal, tambien los patrones de entrenamiento no deben ser iguales ya que estos ocasionan confusión en la red neuronal y provaca que imprima algunos patrones no existentes, por los problemas mencionados anteriormente fue que decidi elegir el tema de simbolos de peligro y elegi simbolos que no se parecian tanto para que fuera a la red mas facil identificar el patron que deaba obtener con los de entrenamiento.

	

\end{document}